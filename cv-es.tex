\documentclass[11pt,a4paper,sans]{moderncv}

\moderncvstyle{casual}                             % style options are 'casual' (default), 'classic', 'oldstyle' and 'banking'
\moderncvcolor{blue}                               % color options 'blue' (default), 'orange', 'green', 'red', 'purple', 'grey' and 'black'
%\renewcommand{\familydefault}{\sfdefault}         % to set the default font; use '\sfdefault' for the default sans serif font, '\rmdefault' for the default roman one, or any tex font name
\nopagenumbers{}                                  % uncomment to suppress automatic page numbering for CVs longer than one page

% character encoding
\usepackage[utf8]{inputenc}                       % if you are not using xelatex ou lualatex, replace by the encoding you are using
%\usepackage{CJKutf8}                              % if you need to use CJK to typeset your resume in Chinese, Japanese or Korean

% adjust the page margins
\usepackage[scale=0.75]{geometry}
%\setlength{\hintscolumnwidth}{3cm}                % if you want to change the width of the column with the dates
%\setlength{\makecvheadnamewidth}{7cm}           % for the 'classic' style, if you want to force the width allocated to your name and avoid line breaks. be careful though, the length is normally calculated to avoid any overlap with your personal info; use this at your own typographical risks...

% personal data
\name{Ricardo}{Münch}
\title{Curriculum Vitae}
\address{El Hatillo, Las Marías, Calle F, Qta. El Cacharro}{Caracas, Venezuela}
\phone[mobile]{+58~(416)~1865}
\phone[fixed]{+58~(212)~963~7813}
\email{munch\_141@hotmail.com}
\homepage{www.linkedin.com/in/ricardo-m\"unch-197b78138/}
\photo[64pt][0pt]{perfil}

% to show numerical labels in the bibliography (default is to show no labels); only useful if you make citations in your resume
%\makeatletter
%\renewcommand*{\bibliographyitemlabel}{\@biblabel{\arabic{enumiv}}}
%\makeatother
%\renewcommand*{\bibliographyitemlabel}{[\arabic{enumiv}]}% CONSIDER REPLACING THE ABOVE BY THIS

% bibliography with mutiple entries
%\usepackage{multibib}
%\newcites{book,misc}{{Books},{Others}}
%----------------------------------------------------------------------------------
%            content
%----------------------------------------------------------------------------------
\begin{document}
%\begin{CJK*}{UTF8}{gbsn}                          % to typeset your resume in Chinese using CJK
%-----       resume       ---------------------------------------------------------
\makecvtitle

\section{Educación}
\subsection{Básica}
\cventry{2000-2011}{Bachiller en Ciencias}{U.E. Colegio Claret}{Caracas, Venezuela}{}{}
\subsection{Avanzada}
\cventry{2011-2018(Esperado)}{Ingeniero en Computación}{Universidad Simón Bolívar}{Caracas, Venezuela}{}{}  % arguments 3 to 6 can be left empty

\section{Experiencia profesional}
\cventry{2017-Presente}{Programador}{iKels Consulting}{Caracas, Venezuela}{Programador de \emph{frontend} y \emph{backend} de aplicaciones Web en Umbraco}{}

\section{Habilidades}
\subsection{Lenguajes de programación}
\cvitemwithcomment{Básico}{Java, SQL, CSS, Ruby, Prolog}{}
\cvitemwithcomment{Intermedio}{C\#, HTML, Python, C/C++, Haskell}{}
\subsection{Bases de datos}
\cvitemwithcomment{Manejadores}{SQL Server, SQLite, PostgreSQL}{}
\cvitemwithcomment{Diseño}{UML, ER-E}{}
\subsection{Frameworks y herramientas de desarrollo Web}
\cvitem{}{ASP.NET, ASP.NET MVC, Umbraco CMS, Django, Bootstrap}
\subsection{Herramientas y entornos de desarrollo}
\cvitem{}{Git, Trello, Linux, Windows, Visual Studio, Visual Studio Code, \LaTeX}
\subsection{Metodologías de desarrollo}
\cvitem{}{Desarrollo ágil (usando Scrum)}

\section{Idiomas}
\cventry{Español}{Competencia nativa}{}{}{}{}
\cventry{Inglés}{Competencia profesional completa}{}{}{}{}

\section{Cursos}
\cvlistitem{Ingeniería de Software I, II y III}
\cvlistitem{Inteligencia Artificial I y II}

\section{Intereses}
\cvitemwithcomment{Música}{Toco guitarra y piano semi-bien con algo de teoría musical}{Autodidacta}
\cvitemwithcomment{Cocinar}{Aprendiendo a cocinar no tan bien}{Autodidacta}

\end{document}